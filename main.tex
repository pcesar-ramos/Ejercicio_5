\documentclass[12pt]{beamer}
\usetheme{Boadilla}
\usepackage[utf8]{inputenc}

% Uncomment the following line to allow the usage of graphics (.jpg, .png, etc.)
%\usepackage{graphicx}

\author{Ejercicios de macroeconomía}
\title{Sobrerreacción del Tipo de Cambio}
%\setbeamercovered{transparent} 
%\setbeamertemplate{navigation symbols}{} 
\logo{Mercado laboral} 
%\institute{} 
\date{2020}
%\subject{} 

% Start the document
% -------------------------%
% -------------------------%
\begin{document}
% -------------------------%
% -------------------------%
\setlength{\parindent}{0cm}

\begin{frame}
\titlepage
\end{frame}

\begin{frame}{Introducción}
El objetivo del modelo es explicar la volatilidad del tipo de cambio nominal respecto a otras variables macroeconómicas y frente a un shock monetario.
Este modelo fue desarrollado por Dornbusch (1980), según la paridad de poder adquisitivo un aumento en la cantidad de dinero genera un aumento en el nivel de precios asi tambien, un aumento del tipo de cambio nominal en el largo plazo. 
\end{frame}

\begin{frame}{Variables del modelo}
Todas las variables estan expresadas en logaritmos.
\begin{itemize}
    \item[$m_t$] Cantidad de dinero.
    \item[$s_t$] Tipo de cambio.
    \item[$p_t$] Nivel de precios domestico. 
    \item[$p_t^*$] Nivel de precios internacional.
    \item[$y_t$] Produccion.
    \item[$y_t^d$] Nivel de demanda
    \item[$y_t^n$] Producto potencial. 
    \item[$i_t$] Tasa de interes nominal.
    \item[$i_t^*$] Tasa de interes nominal internacional.
\end{itemize}  
Las variables con supra \textbf{e} indican las expectativas de las variables. 
\end{frame}

\begin{frame}{Ecuaciones del modelo}

$$m_t - p_t = \psi y_t - \theta i_t$$
$$y^{d}_t = \beta_0 + \beta_1(s_t - p_t + p^*_t) - \beta_2 i_t$$
$$\Delta p_t = \mu (y_t - y^n_t)$$
$$\Delta s_t^e = i_t - i_t ^*$$



\end{frame}

% -------------------------%
% -------------------------%
\end{document}
% -------------------------%
% -------------------------%





